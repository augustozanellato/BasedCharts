\section{Conclusioni}
\subsection{Timeline e conteggio ore di lavoro}
Il progetto è stato iniziato a fine Novembre poco dopo la presentazione, in quel momento sono stati analizzati i
requisiti e abbozzata la \textbf{GUI}.\\
Per quanto riguarda il computo delle ore di lavoro sono state necessarie \textbf{22 ore aggiuntive} rispetto alle
\textbf{50 ore} previste dalla specifica di progetto, dovute principalmente al perfezionamento delle funzionalità e
dell'interfaccia nonché al miglioramento della struttura del codice.\\

\begin{center}
    \begin{longtable}{ |l||r| }
        \hline
        \textbf{Fase progettuale} & \textbf{Ore impiegate} \\
        \hline
        Analisi delle specifiche & 3 \\
        Setup toolchain & 3 \\
        Progettazione e bozza GUI & 6 \\
        Studio del framework Qt & 8 \\
        Progettazione, implementazione e unit testing della struttura dati & 10 \\
        Progettazione e implementazione del modello & 8 \\
        Progettazione e implementazione della gerarchia relativa ai grafici & 10 \\
        Implementazione del salvataggio e caricamento dei dati in JSON & 8 \\
        Sviluppo del wizard di creazione del grafico & 2 \\
        Test e debugging\footnote{Stima molto approssimativa in quanto sono stati distribuiti durante tutto lo sviluppo}
            & 6 \\
        Stesura documentazione & 8 \\
        \hline
        \textbf{Ore Totali} & \textbf{72 ore} \\
        \hline
    \end{longtable}
\end{center}

\subsection{Ambiente di sviluppo}
Nello sviluppo del progetto si è scelto di utilizzare come IDE \textbf{CLion}, un ambiente professionale sviluppato
dalla compagnia JetBrains, essendo a pagamento è stata utilizzata una licenza gentilmente messa a disposizione dagli
sviluppatori mediante il loro \href{https://www.jetbrains.com/community/education/#students}{programma per studenti}.
Questa scelta ha portato inoltre all'uso di \textbf{CMake} come \textit{build automation system} e di \textbf{Ninja}
come \textit{build system} invece dell'accoppiata \textbf{QMake}\footnote{Inoltre QMake sembra essere in via di
abbandono dagli sviluppatori di Qt, con l'ultima versione \textit{major} di Qt (6.0) sono infatti passati ad usare CMake
per gestire le build di Qt stesso}-\textbf{Make}; questa scelta è stata presa anche per prendere confidenza con il tooling
\textit{standard de-facto} utilizzato nello sviluppo in C++.\\
L'utilizzo di CMake ha permesso inoltre l'uso agevole ed integrato con l'IDE di \textit{linter} e \textit{static
analyzers}.\\
Lo sviluppo è avvenuto usando l'ultima versione di \textbf{Qt5 (5.15.2)} \footnote{Fortunatamente questo ha portato solo
ad esattamente due incompatibilità con la versione \textit{target} di Qt, entrambe risolte rapidamente.} e l'ultima
versione di \textbf{Clang (13.0.1)} in esecuzione su \textbf{Arch Linux}; sono stati effettuati anche periodici test
sulla macchina virtuale del corso\footnote{Degno di nota è il fatto che l'installazione di Qt sulla VM non è
completa, mancano molti pacchetti presenti nella distribuzione di default di Qt e soprattutto il pacchetto
\texttt{qt5-default}, indispensabile per far compilare il progetto a \textbf{QMake}}.

\subsection{Compilazione ed esecuzione}
È stata resa possibile la compilazione del progetto anche con \textbf{QMake} mediante file \texttt{.pro}, di seguito
le istruzioni:
\begin{minted}{bash}
    $ qmake
    $ make
\end{minted}
L'esecuzione di questi comandi porta alla compilazione del progetto con destinazione \texttt{./BasedCharts} dove
\texttt{.} è la cartella dove è presente il file \texttt{.pro}, tutti i file intermedi generati dalla toolchain di Qt
e dal compilatore sono posti nella cartella \texttt{./build/qmake}.

\subsection{Note finali}
Lo sviluppo del progetto è stato un'attività molto interessante che mi ha coinvolto molto piacevolmente, credo
andrebbero inseriti più progetti del genere nel curriculum universitario.\\
Le maggiori difficoltà incontrate nello sviluppo sono state dovute alla documentazione lacunosa di Qt, soprattutto per
quanto riguarda l'uso dei \textit{ModelMapper}, si sono presentate diverse occasioni in cui è stato più produttivo
visionare i sorgenti di Qt rispetto a leggere la documentazione.
