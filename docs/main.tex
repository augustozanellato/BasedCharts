\documentclass[a4paper]{article}
\usepackage{fontspec}
\usepackage[italian]{babel}
\usepackage{geometry}
\usepackage{hyperref}
\usepackage[utf8]{inputenc}
\usepackage[T1]{fontenc}
\usepackage{subfiles}
\usepackage{xparse}
\usepackage{graphicx}
\usepackage{float}
\usepackage[table]{xcolor}
\usepackage{longtable}
\usepackage[outputdir=../build/docs]{minted}
\graphicspath{{../images/}}
\setmainfont{texgyrepagella}[
    Extension = .otf,
    UprightFont = *-regular,
    BoldFont = *-bold,
    ItalicFont = *-italic,
    BoldItalicFont = *-bolditalic,
]
\hypersetup{
    unicode=true,
    pdftitle={BasedCharts},
    pdfauthor={Augusto Zanellato},
    linkcolor=cyan,
    linkbordercolor=cyan,
    urlcolor=blue,
}
\geometry{left=2.5cm,right=2.5cm,top=2.5cm,bottom=2.5cm}

\begin{document}{\rowcolors{3}{gray!10!white!90}{white}
\pagenumbering{gobble}
\title{
    \includegraphics[scale=0.8]{images/unipd}\\\bigskip
    BasedCharts\footnote{
        Significato di \textit{Based} (da \href{https://web.archive.org/web/20211229133925/https://www.urbandictionary.com/define.php?term=based}{UrbanDictionary}): A word used when you want to recognize someone for being themselves, i.e. courageous and unique or not caring what others think.
    }\\
    \small Progetto di Programmazione ad Oggetti
}
\author{Augusto Zanellato\\\small2000555}
\date{Anno Accademico 2021/2022}
\maketitle
\clearpage
\tableofcontents
\clearpage
\pagenumbering{arabic}
\subfile{sections/introduzione}
\subfile{sections/architettura}
\subfile{sections/conclusioni}
\end{document}
